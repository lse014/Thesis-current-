\section{Neutron Environment of PET-center cyclotrons}
Cyclotrons are particle accelerators used to speed up light particles (e.g. protons). Figure ?? presents a simplistic representation of a cyclotron under operation. Accelerated particles are directed towards a target, which is placed inside a “target box” located at the end of the collision chamber. When the particles and the target nuclei collide, nuclear reactions take place.


In a PET center, cyclotrons are used to activate radioactive tracers such as F-18. The radioisotope is the end product of proton-O-18 fusion. The nuclear reaction produces F-19, an unstable isotope which decays by neutron emission to become F-18*.

    \begin{equation}
    O-18(p,n)F-18: O-18 + p \rightarrow F-19* \rightarrow F-18*+n
    \end{equation}

Operations of PET cyclotrons generates large amount of radiation. In addition to F-18 neutrons, bremsstrahlung, gamma rays and neutrons of different origin are also present. Bremsstrahlung occurs during acceleration of charged particles. After neutron emission, 18-Fluoride de-excites by means of gamma-ray emission – the sole purpose of its application in PET. In other words, gamma-rays infiltrates the space which surrounds the cyclotron.
Interactions between bremsstrahlung radiation and the cyclotron body, as well as secondary reactions are also sources neutron production [Y. Ogata (2011)]. Inside the target box, for instance, measurements made by [Y. Ogata (2011)] suggests the presence of fast neutrons (>8.7 MeV). Thermalization of neutrons due to scattering off the walls and cyclotron? In other words, the neutron environment surrounding cyclotrons is messy.

The detector in question implements gadolinium (a high Z material) for neutron conversion and silicon for particle detection. Gadolinium is a high Z material and consequently prone to photoelectric gamma-ray interference (i.e. ionization). It is also highly reactive to thermal neutrons but less so for slow and fast neutrons.
The foil is activated upon thermal neutron impact and radiates IC electrons and gamma radiation. Conversion electrons generate a strong signal in the detector. High energy gamma-rays are inherently oblivious to the detector and do not interfere with the electron signal.
Following internal conversion is emission of characteristic x-rays, prominently from K_alpha transitions. Furthermore, gadolinium activation can also be caused by gamma-induced excitation [Kandlakunta 2014?]. Struck by energetic gamma-rays, gadolinium is ionized and attains an excited state from which it relaxes by means of x-ray emission. Thus, x-rays originate not only from neutron capture, but also from background gamma-rays.

The thin semiconductor detector is supposedly more sensitive to x-rays than energetic gamma-rays [Kandlakunt 2012]. In a neutron field the infiltration of gamma-rays can become problematic. With sufficiently high gamma-ray intensity the production of x-rays from gamma induced excitation combined with x-rays from neutron capture may overshadow electron signals.
