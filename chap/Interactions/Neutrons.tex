\section{Neutron Interactions}

A neutron constitute three quarks (udd). The up (u) and down (d) quarks have charge (u: +2/3$e$, d: -1/3$e$) and mass. Together they provide the neutron with mass, energy and net charge equal to zero. Because of its intrinsic properties, a neutron is subject to three of the four fundamental forces; weak, strong and gravitational. Due to the lack of charge neutrons do not interact electromagnetically.
Like all particles with mass and energy, the neutron is affected by gravity. The weak and strong nuclear force operate between quarks and are thus relevant when considering neutrons. The strong force is responsible for holding quarks together, to form nucleons, as well as binding protons and neutrons, to form atomic nuclei. The weak force is liable for fusion reactions as well as radioactive decay of atoms. It is also the driving force of free decaying neutrons, with a lifetime of 15 min. Its force carrier, the W boson, is quite heavy. Because of this the weak interaction typically occurs much slower than other interactions. This is why the free neutron has such a long lifetime in comparison to particles prone to other decay forces (e.g. electromagnetically-decaying pions have lifetime $10^{-16}$s).
Nuclear forces operate only when interacting particles are extremely close to one another, the strong force at $10^{-15}$cm and the weak force closer still at $10^{-18}$cm. These distances are considerably small compared to the infinite range of gravity. Though the gravitational range is large, its strength relative to the strong force is negligible. Strength of the strong force also outweighs the weak, by factor $10^6$. (source? Wiki…)

A neutron may interact with matter in one of five ways:
    \begin{enumerate}[noitemsep]
      \item Elastic scattering
      \item Inelastic scattering
      \item Neutron capture
      \item Nuclear spallation
      \item Nuclear fission
    \end{enumerate}

\subsection{Elastic Scattering}

In particle physics, scattering refers to a particle collision which deflects the particle trajectories. After scattering the particles propagate in a different direction than before. For the process to be classified as elastic scattering the systems total kinetic energy and momentum (in the laboratory frame of reference) must be conserved.
Fig ?? illustrates a neutron on a path colliding with a target nucleus at rest. The neutron has kinetic energy $E_N$  and is heading towards the nucleus of atomic number A. During collision, kinetic energy is transferred from the neutron to the target. Elastic collisions conserve kinetic energy, energy lost by the projectile is gained by the target. After collision the recoil nucleus has energy $E_R$ and propagates θ to initial path. Recoil target gains kinetic energy $\Delta E = E_N-E_R$ and is scattered $\phi$. Angles of final trajectories must conserve momentum.
Using classical momentum and kinetic energy equations, combined with a little math, maximum nucleus recoil energy can be expressed as
      \begin{equation}
      \label{eqn:maxRecoil}
      E_R=E_N \left (\frac{A-1}{A+1}  \right )^2
      \end{equation}
and maximum energy transfer as
      \begin{equation}
      \label{eqn:maxTransfere}
      \frac{\Delta E}{E_N} =1- \left (\frac{A-1}{A+1}  \right )^2
      \end{equation}
From equation \ref{eqn:maxTransfere} it is apparent that maximum recoil energy is highest for nuclear targets of low atomic weight.


\subsection{Inelastic Scattering}
Similar to elastic scattering, inelastic scattering results in deflection of neutron trajectory and a recoil nucleus. In inelastic scattering, the target nucleus absorbs and reemits the neutron. In the process some kinetic energy is converted into nuclear excitation. Neutron absorption raises the nucleus to an excited energy state. The nucleus emits one or more gamma rays until it becomes stable. Angular momentum is conserved, however, kinetic energy is not.


\subsection{Neutron Capture}
In neutron capture an incident neutron is absorbed by the target nucleus. Nuclear atomic number increases upon absoption by one unit and consequencly undergoes nuclear compound formation. The resultant compound nucleus is either stable or unstable. The latter leads to emission of nuclear radiation, heavy ion and/or other fundamental particles. (source? Wiki?)
There are two categories of neutron capture, radiative and nonradiative. Radiative capture reactions produce an excited compound nucleus prone to de-excitation via gamma decay; one or more gamma rays are emitted until the nucleus reaches ground state.
On the other hand, nonradiative capture reactions produce a daughter nucleus accompanied by more than one neutron (since single neutrons emission reactions are considered as inelastic scattering) or charged particles like protons and alphas.
Some of the significant capture reactions for neutron detection are B-10(n,a)Li*, Li-7(n,a)H-3 and He-3(n,p)H.
Reaction equations and nomenclature of capture reactions are listed in Table ??.

The importance of capture reactions are discussed in subsection ?? and an in-depth discussion of gadolinium neutron capture is provided in subsection ??.

\subsection{Nuclear Spallation}
...
\subsection{Nuclear Fission}
Neutron induced fission is a nuclear reaction (or nuclear decay process) in which an atom absorbs a neutron and splits into two or more fission fragments. In contrast to nonradiative capture where reaction products are a heavy ion accompanied by smaller particles, fission products are comparable in size, though slightly different. Some reactions also produce neutrons which can contribute to self-sustainable fission chains.
Fission reactions release a tremendous amount of energy compared to other neutron reactions. Because of this they provide an effective method for detector neutrons, especially in high gamma environments, which becomes apparent in later chapters.
