\section{Photon Interactions}
In matter, there many processes in which photons may partake. In radiaiton detection the three most important are photoelectric absorption, Compton scattering and pair production. These processes result in photon absorption and/or scattering.

\subsection{Photoelectric Absorption}
The process of photoelectric absorption involves a photon and an atomic electron. Upon collision the photon is absorbed, and a photoelectron is emitted from the atom. Energy of the disappearing photon is transferred to the electron as kinetic energy and equals $E_{gamma}-E_{bi,m}$
where $E_{bi,m}$ is the binding energy of the photoelectrons original shell. Emission of inner shell electrons (e.g. from K-shell) are most probable.
A vacancy appears in one of the atoms bound shells. This vacancy if filled by free electrons or electrons from other atomic shells. Electron rearrangement in the atom results in x-ray emission. An electron descending from atomic shell m to n is mediated by an x-ray photon of energy

    \begin{equation}
    E_{XR}=E_{bi,m}-E_{bi,n}
    \end{equation}

where $E_{bi,n}$ represents original binding energy the descending electron.
As can be seen in fig??, photoelectric absorption probability decreases with increasing energy. It is the most likely interaction for photons less energetic than 1 MeV. It also depends on atomic number and occurs more frequent in high Z-materials.


\subsection{Compton Scattering}
Another way photons may interact is with free or loosely bound atomic electrons. This process is called compton scattering. Compton, because it was discoved by Arthur Compton and "scattering" because particle trajectories (?) are altered. In the shadow of the speed of EM waves, electron speed is assumed to be negligeble and equal to zero. From the electrons frame of reference the incomming photon has energy $E_{gamma}$. Once in close proximity (?) the photon and electron interacts through (?). During the interaction, energy is transfered from the photon to the electron. The interaction must occur in such a way that laws of convservation are conserved. Energy gained by the electron must equal energy lost by the photon. Also, final particle trajectories must conserve momentum. Assuming the photon travels along a horizontal line and the electron is at rest, the systems initial momentum is represented by the photons inital energy and direction. The net momentum vector, both before and after collision,  is parallel to the horizon and has no vertical component. Thus the scattered particles must travel vertially in opposite directions, but in the same horizontal direction as the initial photon. The scattered trajectory angles must be such that the vertical components add to zero and horizontal components equals initial momentum.
Compton scatter occur in the energy range $10^2eV$ to 100GeV and is the most probable interaction for photons in proximity of $10^4eV$.



\subsection{Pair Production }
pair production requires considerably higher photon energies than the aforementioned. The concept behind pair production is the creation of anti-particles. A highly energetic photon can decay(?) into a particle pair in which the particles are the anti-particle of the other. The lowest energy required is 1.022 MeV. Decay of such a photon results in an electron-positron pair. The pair contributes to charge conservation (net charge is zero) as well as conservation of energy and momentum. Rest energy of electrons/positrons is 1.022/2. Hence a minimum of 1.022 MeV is required for the production. Excess energy beyond this value is gained by the particles as kinetic energy. For higher energies other anti-particle pairs may also be produced, such as muon-antimuon or proton-antiproton [wiki].

\subsection{Beam Attenuation}
The following scenario is illustrated in figure ??. A collimated photon beam of intensity $I$ is directed at an absorbing target material. The beam consists of many individual photons travelling parallel to each other. Some of the photons interact with matter and are either deflected or absorbed. An absorbed photon disappears completely. Deflected photons diverge from their original path, usually at a notable deflection angle. As the beam penetrates the target, constituent photons are removed and beam intensity decreases.

Like most radiation, gamma attenuation can be modeled by Beer-Lamberts law
    \begin{equation}
    I=I_0 e^{-μt}
    \end{equation}
where $I$ is transmitted beam intensity after traversing an absorber of thickness $t$. It is also dependent on the attenuation coefficient $\mu$ of the target, which in turn depends on the target’s density.
