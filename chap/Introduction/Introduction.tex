\chapter{Introduction}
\label{chap:Intro}
He-3 filled proportional counters have long been the working horse in the field of neutron detection. However, in the past decade He-3 has become both scars and costly while the popularity of neutron detectors is increasing. This has inspired the research community to investigate alternative neutron detection methods. A variety of particle detectors and converter material combinations have been proposed as replacements of the He-3 counters. Of particular interest is the use of semiconductors.

In comparison to gaseous detectors, semiconductors have thin sensitive layers of high stopping power and ionizing particles can be detected with relatively small dimensions.
Their benefits are not limited to small detector volumes, semiconductors also operate at lower voltages and provide excellent energy and position resolution for charged particles. Some drawbacks are their limitations to small sizes due to high costs per surface area and susceptibility to performance degradation from radiation-induced damage.

Out of many possible neutron sensitive materials, Gadolinium shows promising characteristics, with the highest known thermal neutron absorption cross-section and high reaction energy.

….

\section{About the Thesis}
The main goal of this thesis is to study neutron detection abilities of an ALPIDE detector coupled with a gadolinium-based neutron convert foil. The detector layout consists of one ALPIDE chip and a thin sheet of natural gadolinium, placed closely in front of the chips sensitive area. The ALPIDE is a semiconductor-based particle detector and is blind to neutrons. The gadolinium foils role is converting incoming thermal neutrons into charged particles such that the ALPIDE subsequently may detects reaction products indicating neutron presence. The reaction products of thermal neutron capture in gadolinium are mainly prompt gamma-rays and electrons. The ALPIDEs development was motivated by the upgrade of the Inner Tracking System of ALICE and thus its main focus is detecting minimum ionizing particles (MIP). Its thin semiconductor geometry makes it nearly impervious to gamma radiation, thus the principle signal generator are the electrons. However, if the presence of gamma-rays is significant the accumulation of induced signals may no longer be neglectable. The ability to discriminate gamma and electron induced signals is an important characteristic of neutron detectors.

It is therefore imperative that the ALPIDE ….

\section{Thesis Outline}
...
\section{Citation Principles}
...
