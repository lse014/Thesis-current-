\chapter{The ALPIDE Sensor}

Alice Pixel Detector (ALPIDE) is a monolithic active pixel sensor (MAPS) based on TowerJazz 180nm CMOS technology. The sensor was developed for the upgrade of the Inner Tracking System (ITS) of the ALICE experiment at CERN. High particle rates and densities makes up the ITS environment. Not only capable of handling such environments, the ALPIDE shows great versatility and is also applicable outside of high-energy physics.
The sensor with its exceptional properties has sparked interests thereof in the field of medical physics. A local pCT research group in Bergen at the Institute of Physics and Technology (IFT), Bergen University is currently working on the design of a proton CT implementing the ALPIDE chip in a…

\section{Chip Architecture}
ALPIDE is realized on a silicon substrate on which a highly resistive epitaxial layer (sensor active volume) is grown. A potential barrier forms where the heavy doped (P++) substrate and lightly doped (P-) epitaxial layer meet. Electrons (e) are vertically confined by the potential barriers and diffuse laterally across pixels. N-well diodes are the sensing elements and are surrounded by the depletion volume. Moderate reverse bias can be applied to the substrate to increase depletion volume and improve charge collection.
An important feature is the implementation of a deep p-well which shields n-wells of pMOS transistor from the epitaxial layer. This prohibits diodes and n-wells from competing in charge collection.
On top of the chip are up to six aluminum metal layers suitable to implement high density and low digital circuits [Abelev 2014].
One ALPIDE chip holds 512x1024 (row x column) pixels and baseline dimensions are 15x30 cm$^2$ [Abelev 2014]. Pixels are grouped into
Possible thickness of the epitaxial layer ranges from 18um to 30um. Each pixel (28x28 um$^2$) embodies a diode, a front-end amplifier and shaping stage, a discriminator and a digital section [G. Aglieri].



...


%new 23/07/2020

One ALPIDE chip is fabricated on a single slab of silicon and holds 512x1024 (row x column) pixels. Baseline dimensions of a chip are 15 cm x 30 cm [Abelev 2014]. A pixel is merely confined to a cubic volume in which all necessary pixel components, diode and readout electronics, are included. In a chip, there are no physical boundaries separating the active volume of pixels. Figure ?? illustrates a three-dimensional cut-out of ALPIDE with four pixels.

Each pixel (28x28 um$^2$) embodies a diode, a front-end amplifier and shaping stage, a discriminator and a digital section [G. Aglieri]. A cross sectional view of a pixel is illustrated in fig ??. The sensor is realized on a silicon substrate on which a highly resistive epitaxial layer (sensor active volume) is grown. Into the epitaxial layer p-wells are placed. A potential barrier forms where the heavily p-doped (P++) substrate and (P+) p-wells meet the lightly p-doped (P-) epitaxial layer. Electrons (e) are vertically confined by the potential barriers and diffuse laterally across pixels.

The ALPIDE chip is based on the 180 nm CMOS technology of TowerJazz. An important design feature is the deep p-well which shields n-wells of pMOS transistor from the active layer. This prohibits diodes and n-wells from competing in collecting electrons. The feature allows the full use of CMOS circuitry, including the pMOS transistors, in the epitaxial layer without imparing charge collection [S. Kushpil (2017)].

N-well diodes are the sensing elements and are surrounded by the depletion volume. Moderate reverse bias can be applied to the substrate in order to increase depletion volume and improve charge collection. [S. Kushpil (2017)]
Most part of the epitaxial layer is free of electric field. Charge is left to thermally diffuse in the active volume until collected by the diode or it recombines with the atomic structure. Because of this MAPS have slow collection times, approximately 100 ns. [*]

On top of the chip are up to six aluminum metal layers suitable to implement high density and low digital circuits [Abelev 2014]. Possible thickness of the active layer ranges from 18 um to 30 um. The diode has a diameter ~2 um, approximately 100 times smaller than pixel area.


\section{Radiation Sensitivity}
For the tracking of charged particles, ALPIDE exhibits an impressively large detection efficiency (>99\%), a low fake hit probability (<10$^{-5}$ pixel $^-1$ event $^{-1}$) and spatial resolution (≈5 um) [F. Colamaria].
In theory, because of its thin active volume, ALPIDE should be nearly oblivious to crossing photons, especially in the presence of particle radiation of which detection efficiency is comparatively large. It has been shown that ALPIDE sensitivity to 662 keV photons measures below 1 per mil (‰)[F. Colamaria]. As photon interaction becomes less probable for higher energies (as discussed in section ??) so does their probability of being detected. In light of the particle detection efficiency, photon signal contributions are of minor importance, with the exception of measurements in highly intense photon environments, where signal accumulation can become problematic.
