\chapter{The ALPIDE Sensor}

Alice Pixel Detector (ALPIDE) is a monolithic active pixel sensor (MAPS) based on TowerJazz 180nm CMOS technology. The sensor was developed for the upgrade of the Inner Tracking System (ITS) of the ALICE experiment at CERN. High particle rates and densities makes up the ITS environment. Not only capable of handling such environments, the ALPIDE shows great versatility and is also applicable outside of high-energy physics.
The sensor with its exceptional properties has sparked interests thereof in the field of medical physics. A local pCT research group in Bergen at the Institute of Physics and Technology (IFT), Bergen University is currently working on the design of a proton CT implementing the ALPIDE chip in a…

ALPIDE is realized on a silicon substrate on which a highly resistive epitaxial layer (sensor active volume) is grown. A potential barrier forms where the heavy doped (P++) substrate and lightly doped (P-) epitaxial layer meet. Electrons (e) are vertically confined by the potential barriers and diffuse laterally across pixels. N-well diodes are the sensing elements and are surrounded by the depletion volume. Moderate reverse bias can be applied to the substrate to increase depletion volume and improve charge collection.
An important feature is the implementation of a deep p-well which shields n-wells of pMOS transistor from the epitaxial layer. This prohibits diodes and n-wells from competing in charge collection.
On top of the chip are up to six aluminum metal layers suitable to implement high density and low digital circuits [Abelev 2014].
One ALPIDE chip holds 512x1024 (row x column) pixels and baseline dimensions are 15x30 cm$^2$ [Abelev 2014]. Pixels are grouped into
Possible thickness of the epitaxial layer ranges from 18um to 30um. Each pixel (28x28 um$^2$) embodies a diode, a front-end amplifier and shaping stage, a discriminator and a digital section [G. Aglieri].
